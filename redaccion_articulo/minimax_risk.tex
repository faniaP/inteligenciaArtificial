\documentclass[12pt,twocolumn,draft]{article} 

\usepackage{times}
\usepackage{graphicx}
\usepackage{amssymb}
\usepackage{url,hyperref}
\usepackage{cite}
\usepackage{algorithm}
\usepackage{algorithmic}
\floatname{algorithm}{Pseudoc\'odigo}
\renewcommand{\listalgorithmname}{Lista de algoritmos}
\renewcommand{\algorithmicrequire}{\textbf{Entrada:}}
\renewcommand{\algorithmicensure}{\textbf{Salida:}}
\renewcommand{\algorithmicend}{\textbf{fin}}
\renewcommand{\algorithmicif}{\textbf{si}}
\renewcommand{\algorithmicthen}{\textbf{entonces}}
\renewcommand{\algorithmicelse}{\textbf{si no}}
\renewcommand{\algorithmicelsif}{\algorithmicelse,\ \algorithmicif}
\renewcommand{\algorithmicendif}{\algorithmicend\ \algorithmicif}
\renewcommand{\algorithmicfor}{\textbf{para}}
\renewcommand{\algorithmicforall}{\textbf{para todo}}
\renewcommand{\algorithmicdo}{\textbf{hacer}}
\renewcommand{\algorithmicendfor}{\algorithmicend\ \algorithmicfor}
\renewcommand{\algorithmicwhile}{\textbf{mientras}}
\renewcommand{\algorithmicendwhile}{\algorithmicend\ \algorithmicwhile}
\renewcommand{\algorithmicloop}{\textbf{repetir}}
\renewcommand{\algorithmicendloop}{\algorithmicend\ \algorithmicloop}
\renewcommand{\algorithmicrepeat}{\textbf{repetir}}
\renewcommand{\algorithmicuntil}{\textbf{hasta que}}
\renewcommand{\algorithmicprint}{\textbf{imprimir}} 
\renewcommand{\algorithmicreturn}{\textbf{devolver}} 
\renewcommand{\algorithmictrue}{\textbf{verdadero }} 
\renewcommand{\algorithmicfalse}{\textbf{falso }}



\begin{document}

\title{\textbf{Implementaci\'on de Risk con algoritmo de Minimax.}}
\author{Prieto, Estefan\'ia$^{[1]}$\\
Galicia, Fernando$^{[1]}$\\
Galv\'an, Antonio$^{[1]}$\\
$^{[1]}$Facultad de Ciencias, U.N.A.M.\\
\\
estefaniaprieto@ciencias.unam.mx\\
fernandogamen@ciencias.unam.mx\\
g.antonio@ciencias.unam.mx\\
\date{21-Noviembre-2014}
}

\twocolumn[\begin{@twocolumnfalse}
\maketitle
\thispagestyle{empty}
\begin{abstract}
A diferencia de los juegos de apuesta, donde el jugador se pregunta \textit{"?`Cu\'al es la mejor jugada para ganar un juego?"} y as\'i poder ser el due\~{n}o de un premio (generalmente un incentivo monetario), es bien sabido en la teor\'ia de juegos la motiva escenarios tales c\'omo el ajedrez, go, gato, etc. No existe tal pregunta, si no, \'esta se replantea una expresi\'on de la forma \textit{"?`Existe una mejor forma de jugar en tal escenario?"}.\\
Por lo cu\'al se propone un modelo de Inteligencia Artificial para una versi\'on acotada del juego \textbf{Risk} basado en minimax, con base en estrategias muy complicadas implementadas por un experto, hasta muy b\'asicas diseñadas por un novato en el juego.\\
\end{abstract}
\end{@twocolumnfalse}]


\section{Introducci\'on.}

Hacer ac\'a la introducci\'on de minimax.


\section{Juegos con informaci\'on perfecta. [MODIFICAR A INFORMARCI\'ON IM-PERFECTA Y ARGUMENTAR EL POR QUE PODEMOS ADAPTARLO AS\'I.]}
Explicar por que catalogamos al \textbf{RISK} c\'omo un juego de informaci\'on perfecta y por que hemos elegido esta implementaci\'on.


% % % % % % % % % % % % % % % % % % % % % % % % % % % % % % % % % % % % % % % % % % % % % % % %
% % % % % % % % % % % % RISK ACOTADO % % % % % % % % % % % % % % % % % % % % % % % % % % % % % %
% % % % % % % % % % % % % % % % % % % % % % % % % % % % % % % % % % % % % % % % % % % % % % % % 
\section{Risk acotado.}

Tal y c\'omo se plantea en el juego original (\textbf{\textit{ve\'ase \cite{RISK}}}) el objetivo del juego continua siendo la dominaci\'on total de un territorio dado, de tal forma
que el juego queda concluido cu\'ando todos los territorios quedan bajo la dominaci\'on de 
un jugador.\\
En esta implementaci\'on acotaremos la cantidad de continentes, es decir, el desarrollo sera unicamente en un solo continente, tambi\'en la cantidad de dados se ve acotada a unicamente dos dados y restringido a dos jugadores.\\

Sin embargo mantendremos las dem\'as condiciones iniciales con respecto a las tropas y al equivalente de tropas en cada territorio, es decir:
\begin{list}{*}{}
\item Cada unidad representa una \textit{Armada}.
\item Cada \textit{Caballer\'ia} representa 5 unidades.
\item Cada \textit{Artiller\'ia} representa 10 unidades.
\end{list}

Teniendo ya esto definido, entonces, cada jugador tendr\'a un ejercito inicial de 35 tropas.

\section{Descripci\'on del agente.}
Aqu\'i es donde describimos el comportamiento del agente.


\section{Funci\'on de evaluaci\'on e implementaci\'on.}
El tablero de juego estar\'a representado por medio de una gr\'afica, en la cual, cada nodo  ser\'a el representante de un pa\'is, \'estos comparten frontera con aquellos nodos en los cuales existe una arista que los conecta.\\

Para introducir la funci\'on de evaluaci\'on primero observaremos el siguiente conjunto de definiciones:\\

\textbf{\textit{Definici\'on:}}
Definimos a la funci\'on  \textit{"tropas"} denotadas como 
$t(i,j)$ de la siguiente manera:
\begin{center}
$t(i,j)$ \textit{:= N\'umero de tropas del i-\'esimo  jugador en el j-\'esimo nodo}.\\
\end{center}

\textbf{\textit{Definici\'on:}}
Definimos al conjunto de nodos pertenecientes a la gr\'afica y los denotamos c\'omo $\nu$ y al conjunto de nodos del i-\'esimo jugador c\'omo $\nu_{i}$\\

Y nos permitiremos sobrecargar la notaci\'on definiendo la funci\'on auxiliar $J(\nu)$ que tiene c\'omo objetivo indicar el jugador asociado al nodo $\nu$.\\


Con estas observaciones nos permitimos definir ahora a la funci\'on de evaluaci\'on c\'omo sigue:\\

$$ 
F_{eval}:= \left\{\begin{array}{c l}
  \infty\ \textit{Max\ resulta\ ser\ ganador.}\\
  -\infty\ \textit{Min\ resulta\ ser\ ganador.}\\
  ((\sum_{i=1}^{\nu} t(i,j)) \qquad \qquad \qquad\\
  +c_{1}+c_{2}+\nu_{i}-\nu_{j}) \qquad \textit{E.O.C.}
\end{array}
\right.
$$

Donde definimos a la funci\'on $c_{1}$ de la siguiente manera:

\begin{algorithm}
\begin{algorithmic}[1]
\ENSURE Determinar la imagen de la funci\'on $c_{1}$
\STATE{$cuenta = 0$}
\FORALL{$\nu^{\prime}$ $\in \nu_{1}$}
\FORALL{$u$ $\in$ \textit{Vecinos}($\nu^{\prime}$)}
\IF{$u \ \in \nu_{2}$}
\STATE{$cuenta = cuenta + 1$}
\ENDIF
\ENDFOR
\IF{$cuenta \ge 1$}
\STATE{$-1$}
\ELSE
\STATE{1}
\ENDIF
\ENDFOR
\RETURN $cuenta$

\end{algorithmic}
\caption{Definici\'on de la funci\'on $c_{1}$}
\label{c1}
\end{algorithm}

El algoritmo~\ref{c1} muestra el comportamiento de de $c_{1}$ sea cual quiera que sea.\\

Continuamos por definir la funci\'on $c_{2}$ as\'i de esta manera observemos que:\\

\begin{algorithm}
\begin{algorithmic}[1]
\ENSURE Determinar la imagen de la funci\'on $c_{2}$
\STATE{$cuenta = 0$}
\FORALL{$\nu$ $\in$ \textit{Vecinos}($\nu$)}
\IF{$J(\nu) \neq \emptyset$}
\STATE{$J()$}
\ENDIF
\ENDFOR


\end{algorithmic}
\caption{Definici\'on de la funci\'on $c_{2}$}
\label{c2}
\end{algorithm}

\section{Especificaciones del programa.}
Aqu\'i van las especificaciones de c\'omo es que programamos nuestro \textbf{RISK}, lenguaje, resultados, etc.

\section{Conclusiones.}
Informar las conclusiones que hemos encontrado en nuestra implementaci\'on. \cite{2personGT}


% %Toda la bibliografía consultada debe de estar anexada en el archivo
% %referencias.bib 
\newpage
\bibliographystyle{plain}	
\bibliography{referencias.bib}{}

\end{document}